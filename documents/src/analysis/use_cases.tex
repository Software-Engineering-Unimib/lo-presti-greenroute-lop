\newtcolorbox{usecasebox}[1][]{colback=gray!10,colframe=black, #1}

\subsection{Casi d’uso in formato dettagliato}
\begin{usecasebox}[title=Autenticati]
	\textbf{Ambito:}
	GreenRoute \\

	\textbf{Livello:}
	Obiettivo utente \\

	\textbf{Attore primario:}
	Persona \\

	\textbf{Stakeholder e interessi:} \\
	La persona desidera dimostrare la propria identità per accedere alle funzionalità del sistema riservate agli utenti autenticati \\

	\textbf{Precondizioni:}
	La persona non è autenticata \\

	\textbf{Postcondizioni:}
	La persona (ora utente) ha dimostrato la propria identità al sistema ed è stata riconosciuta \\

	\textbf{Scenario di successo principale:}
	\begin{enumerate}[itemsep=0pt, topsep=0pt]
		\item La persona richiede di autenticarsi al sistema
		\item Il sistema chiede alla persona di fornire il proprio indirizzo email e la password
		\item La persona inserisce le credenziali richieste
		\item Il sistema conferma che la persona è stata autenticata con successo e fornisce informazioni dell’account, come lo username
	\end{enumerate}
	\vspace{1\baselineskip}

	\textbf{Estensioni:} \\
	\hspace*{1em} 4a.1 Il sistema informa la persona che l’autenticazione non è riuscita e offre la \\
	\hspace*{3em} possibilità di registrare un nuovo utente o riprovare \\[0.5em]
	\hspace*{2em} 4a.2a La persona sceglie di riprovare \\[0.5em]
	\hspace*{2em} 4a.3a Il caso d’uso riparte dal passo 2 \\[0.5em]
	\\
	\hspace*{2em} 4a.2b La persona sceglie di registrare un nuovo utente \\[0.5em]
	\hspace*{2em} 4a.3b Ha inizio il caso d’uso \textbf{Registrazione utente} \\[0.5em]

	\textbf{Requisiti speciali:} Nessuno \\

	\textbf{Elenco di variazioni tecnologiche e dei dati:} Nessuna \\

	\textbf{Frequenza di occorrenza:}
	Potenzialmente quasi continua \\

	\textbf{Questioni aperte:} Nessuna \\
\end{usecasebox}
\newpage

\begin{usecasebox}[title=Registra un utente]
	\textbf{Ambito:}
	GreenRoute \\

	\textbf{Livello:}
	Obiettivo utente \\

	\textbf{Attore primario:}
	Persona \\

	\textbf{Stakeholder e interessi:} \\
	La persona desidera registrare la propria identità nel sistema per accedere alle funzionalità riservate agli utenti autenticati \\

	\textbf{Precondizioni:}
	La persona non è autenticata  \\

	\textbf{Postcondizioni:}
	La persona ha registrato un’identità nel sistema \\

	\textbf{Scenario di successo principale:}
	\begin{enumerate}[itemsep=0pt, topsep=0pt]
		\item La persona richiede di registrare un’identità nel sistema
		\item Il sistema chiede alla persona di fornire un indirizzo email e di inserire la stessa password due volte
		\item La persona inserisce le credenziali richieste
		\item Il sistema invia alla persona un’email di verifica
		\item La persona verifica il proprio indirizzo email entro il tempo consentito
		\item Il sistema conferma che l’utente è stato registrato con successo
	\end{enumerate}
	\vspace{1\baselineskip}

	\textbf{Estensioni:} \\
	\hspace*{1em} 4a.1 Il sistema informa la persona che l’email fornita non è valida \\[0.5em]
	\hspace*{1em} 4a.2 Il caso d’uso riparte dal passo 2 \\[0.5em]
	\\
	\hspace*{1em} 4b.1 Il sistema informa la persona che le due password fornite non coincidono \\[0.5em]
	\hspace*{1em} 4b.2 Il caso d’uso riparte dal passo 2 \\[0.5em]
	\\
	\hspace*{1em} 4c.1 Il sistema informa la persona che le password fornite non sono valide \\[0.5em]
	\hspace*{1em} 4c.2 Il caso d’uso riparte dal passo 2 \\[0.5em]
	\\
	\hspace*{1em} 5a.1 La persona tenta di verificare l’email dopo la scadenza del link di verifica \\[0.5em]
	\hspace*{1em} 5a.2 Il sistema segnala l’errore e invita la persona a registrarsi nuovamente \\[0.5em]

	\textbf{Requisiti speciali:}
	\begin{itemize}[label=-,itemsep=0pt, topsep=0pt]
		\item Il sistema deve imporre regole sulle password consentite, garantendo che siano almeno moderatamente sicure
	\end{itemize}
	\vspace{1\baselineskip}

	\textbf{Elenco di variazioni tecnologiche e dei dati:} Nessuna \\

	\textbf{Frequenza di occorrenza:}
	Potenzialmente quasi continua \\

	\textbf{Questioni aperte:} Nessuna \\
\end{usecasebox}
\newpage

\begin{usecasebox}[title=Visualizza i percorsi possibili]
	\textbf{Ambito:}
	GreenRoute \\

	\textbf{Livello:}
	Obiettivo utente \\

	\textbf{Attore primario:}
	Persona \\

	\textbf{Stakeholder e interessi:} \\
	La persona desidera che il sistema fornisca raccomandazioni di percorso adattate alle proprie esigenze \\

	\textbf{Precondizioni:} Nessuna \\

	\textbf{Postcondizioni:}
	Il sistema ha mostrato alla persona i percorsi che soddisfano le sue richieste \\

	\textbf{Scenario di successo principale:}
	\begin{enumerate}[itemsep=0pt, topsep=0pt]
		\item Il sistema fornisce alla persona le informazioni della mappa
		\item La persona richiede di visualizzare i percorsi verso una destinazione specifica
		\item Il sistema chiede di fornire una posizione di partenza
		\item La persona fornisce la posizione di partenza
		\item Il sistema mostra le modalità di trasporto disponibili e chiede di selezionarne un sottoinsieme
		\item La persona seleziona le modalità di trasporto da consentire
		\item Il sistema chiede se i percorsi debbano privilegiare la velocità, l’ecosostenibilità o un compromesso tra le due
		\item La persona seleziona un’opzione
		\item Il sistema fornisce un elenco di percorsi possibili con stime di tempo, distanza percorsa ed emissioni di CO\textsubscript{2}
	\end{enumerate}
	\vspace{1\baselineskip}

	\textbf{Estensioni:} Nessuna \\

	\textbf{Requisiti speciali:} Nessuno \\

	\textbf{Elenco di variazioni tecnologiche e dei dati:}
	\begin{itemize}[label=-,itemsep=0pt, topsep=0pt]
		\item La persona deve poter inserire le posizioni di partenza e arrivo sia manualmente sia utilizzando la posizione GPS corrente del dispositivo
	\end{itemize}
	\vspace{1\baselineskip}

	\textbf{Frequenza di occorrenza:}
	Potenzialmente quasi continua \\

	\textbf{Questioni aperte:} Nessuna \\
\end{usecasebox}
\newpage

\subsection{Casi d’uso in formato breve}

\paragraph{Gestisci la cronologia dei percorsi:}
Gli utenti possono visualizzare la cronologia dei percorsi effettuati e hanno la possibilità di eliminare singole voci, sezioni o l’intera cronologia.\\

\paragraph{Gestisci gli amici:}
Gli utenti possono visualizzare l’elenco di tutti i propri amici. Per ciascun amico possono rimuovere l’amicizia, invitarlo a una sfida di gioco, visualizzarne gli attributi di gioco o bloccarlo.\\

\paragraph{Aggiungi un amico:}
Gli utenti possono inviare richieste di amicizia ad altri utenti specificando il loro username. Il destinatario può accettare la richiesta, rifiutarla o bloccare il mittente.\\

\paragraph{Gestisci gli utenti bloccati:}
Gli utenti possono visualizzare l’elenco di tutti gli utenti bloccati nel tempo e possono sbloccarne uno o più qualsiasi.\\

\paragraph{Visualizza lo stato di gioco:}
Gli utenti possono visualizzare tutti i propri attributi di gioco, come punti accumulati, sfide vinte e badge ottenuti, nonché consultare classifiche che li confrontano con i propri amici.\\

\paragraph{Gestisci il sistema:}
Gli amministratori possono accedere a informazioni di sistema non disponibili agli utenti standard, come l’utilizzo delle API, ed eseguire azioni a livello globale, ad esempio bloccare temporaneamente la creazione di nuovi account.\\

\paragraph{Gestisci gli amministratori:}
Gli amministratori possono gestire i permessi degli amministratori di livello inferiore, assegnando privilegi amministrativi di livello inferiore o revocandoli completamente.\\
