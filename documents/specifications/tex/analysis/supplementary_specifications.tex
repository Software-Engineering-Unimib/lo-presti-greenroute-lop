This document is the repository of all GreenRoute requirements not captured in the use cases.\\

{\large \textbf{Functionality}} \\
\begin{itemize}[label=-,itemsep=0pt, topsep=0pt]
	\item All sensitive user data should be handled and stored safely and privately. Encryption is a must over networks
	\item The system should impose rules on what passwords are allowed that ensure all passwords are at least moderately secure
	\item Access to the system must be possible through a mobile app that is available on both android and IOS
\end{itemize}
\vspace{1\baselineskip}

{\large \textbf{Usability}} \\
\begin{itemize}[label=-,itemsep=0pt, topsep=0pt]
	\item The interaction flow must follow standard mobile application usability conventions familiar to users of comparable systems
	\item For operations that take more than a second, the system should visually display that it is not stuck, but is actively working on responding to the input
\end{itemize}
\vspace{1\baselineskip}

{\large \textbf{Reliability}} \\
\begin{itemize}[label=-,itemsep=0pt, topsep=0pt]
	\item The system should handle access to persistent storage safely and robustly, taking into account race conditions and eventual crashes
\end{itemize}
\vspace{1\baselineskip}

{\large \textbf{Performance}} \\
\begin{itemize}[label=-,itemsep=0pt, topsep=0pt]
	\item System response times should be minimized, and interactions should feel as instantaneous as possible, within the limits of the hardware used and of what the available development resources allow
\end{itemize}
\vspace{1\baselineskip}

{\large \textbf{Supportability}} \\
\begin{itemize}[label=-,itemsep=0pt, topsep=0pt]
	\item The technologies used to develop the app must enable building for both Android and iOS from a single codebase, while minimizing the need for testing on both platforms
	\item The system should be implemented using high-level, high-abstraction programming languages and frameworks to maximize development speed and ease of maintenance
\end{itemize}
