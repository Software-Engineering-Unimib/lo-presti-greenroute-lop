\documentclass{article}

\usepackage[margin=1in]{geometry}
\usepackage{fancyhdr}
\usepackage{titlesec}
\usepackage[table]{xcolor}
\usepackage{tikz}
\usepackage[most]{tcolorbox}
\usepackage{enumitem}
\usepackage{pgfgantt}
\usepackage[italian]{babel}

\usetikzlibrary{positioning}
\setlength{\parindent}{0pt}  % disabilità l'indentazione così che possa essere utilizzata per rendere il codice leggibile
\hbadness=10000

\pagestyle{fancy}
\fancyhf{}
\fancyhead[L]{Gestione di GreenRoute}
\fancyhead[R]{\thepage}

\titleformat{\section}{\large\bfseries}{\thesection.}{1em}{}
\titleformat{\subsection}{\normalsize\bfseries}{\thesubsection.}{1em}{}

\title{Gestione di GreenRoute}
\author{Fabio Lo Presti}
\date{}

\begin{document}	
	\maketitle
	\thispagestyle{empty}
	Questo documento contiene tutti gli artefatti di gestione prodotti per il sistema GreenRoute.
	Consultare l'indice per saltare ad un artefatto specifico.

	\tableofcontents
	\newpage

	\section{Sforzo richiesto per le task e dipendenze}
	\begin{table}[htbp]
	\centering
	\begin{tabular}{ | p{2.5cm} | p{6cm} | p{2.5cm} | p{2.5cm} | }
		\hline
			\rowcolor{gray!20}
			Indice & Nome attività & Impegno (in giorni) & Dipendenze \\
		\hline
			T1 & Identificazione dei requisiti chiave e dei rischi & 6 & Nessuna \\
		\hline
			T2 & Valutazione delle opzioni tecnologiche & 5 & Nessuna \\
		\hline
			T3 & Creazione del piano di progetto iniziale & 3 & T1, T2 \\
		\hline
			T4 & Raffinamento dei requisiti & 6 & T1 \\
		\hline
			T5 & Analisi del dominio & 6 & T1 \\
		\hline
			T6 & Progettazione logica e definizione dei meccanismi chiave & 7 & T4, T5 \\
		\hline
			T7 & Implementazione incrementale dei casi d'uso & 8 & T6 \\
		\hline
			T8 & Integrazione continua e testing & 8 & T7 \\
		\hline
			T9 & Valutazione dell'iterazione e aggiornamento del piano & 2 & T7, T8 \\
		\hline
			T10 & Esecuzione dell'iterazione 2 & 12 & T9 \\
		\hline
	\end{tabular}
\end{table}

	\newpage

	\section{Diagramma di gantt}
	\resizebox{\textwidth}{!}{
	\begin{ganttchart}[vgrid, hgrid, time slot format=isodate]{2026-01-17}{2026-02-15}
		\gantttitle{Gennaio}{15}
		\gantttitle{Febbraio}{15} \\
		
		\gantttitlelist{17,...,31}{1}
		\gantttitlelist{1,...,15}{1} \\
		
		% Fase di ideazione
		\ganttgroup[group/.append style={fill=green!20}]{Ideazione}{2026-01-17}{2026-01-22} \\
		
		\ganttbar{Identificazione dei requisiti chiave e dei rischi}{2026-01-17}{2026-01-22} \\
		\ganttbar{Valutazione delle opzioni tecnologiche}{2026-01-18}{2026-01-22} \\
		\ganttbar{Creazione del piano di progetto iniziale}{2026-01-20}{2026-01-22} \\
		
		% Fase di elaborazione
		\ganttgroup[group/.append style={fill=green!20}]{Elaborazione}{2026-01-23}{2026-02-15} \\
		
		% iterazione 1
		\ganttgroup[group/.append style={fill=gray!30}]{Iterazione 1}{2026-01-23}{2026-02-03} \\

		\ganttbar{Raffinamento dei requisiti}{2026-01-23}{2026-01-28} \\
		\ganttbar{Analisi del dominio}{2026-01-23}{2026-01-28} \\
		\ganttbar{Progettazione logica e definizione dei meccanismi chiave}{2026-01-24}{2026-01-30} \\
		\ganttbar{Implementazione parziale del caso d'uso "Visualizza i percorsi possibili"}{2026-01-27}{2026-02-03} \\
		\ganttbar{Configurazione del integrazione continua e del testing}{2026-01-27}{2026-02-03} \\
		\ganttbar{Aggiornamento del piano }{2026-02-02}{2026-02-03} \\
		
		% iterazione 2
		\ganttgroup[group/.append style={fill=gray!30}]{Iterazione 2}{2026-02-04}{2026-02-15} \\

		\ganttbar{Ulteriore analisi}{2026-02-04}{2026-02-05} \\
		\ganttbar{Refactoring}{2026-02-05}{2026-02-7} \\
		\ganttbar{Ulteriore progettazione}{2026-02-06}{2026-02-08} \\
		\ganttbar{Implementazione completa del caso d'uso "Visualizza i percorsi possibili"}{2026-02-06}{2026-02-13} \\
		\ganttbar{Scrittura di test}{2026-02-06}{2026-02-13} \\
		\ganttbar{Ulteriore refactoring}{2026-02-12}{2026-02-13} \\
		\ganttbar{Scrittura relazione}{2026-02-13}{2026-02-15} \\
	\end{ganttchart}
}

	\newpage

	\section{Stima del rischio}
	\begin{table}[htbp]
	\centering
	\label{tab:risk_probabilities}
	\begin{tabular}{ | p{10cm} | p{2cm} | p{2cm} | }
		\hline
			\rowcolor{gray!20}
			Risk & Probability & Effects \\
		\hline
			Testing the iOS application proves to be more complex than anticipated &  Moderate & Tolerable \\
		\hline
			Ensuring consistent behavior between the iOS and Android versions of the application may be so complex that it is better to maintain separate codebases or target a single operating system
			&  Low & Tolerable \\
		\hline
			The client-side technology may be unsuitable for the graphics-intensive operations required by the system &  Low & Serious \\
		\hline
			The implementation of the first iteration may not be completed before the iteration's deadline &  Moderate & Serious \\
		\hline
	\end{tabular}
\end{table}
	\newpage
\end{document}
