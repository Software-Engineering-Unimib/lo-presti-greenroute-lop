\newtcolorbox{usecasebox}[1][]{colback=gray!10,colframe=black, #1}

\subsection{Fully dressed use cases}
\begin{usecasebox}[title=Authentication]
	\textbf{Scope:}
	GreenRoute \\

	\textbf{Level:}
	User goal \\

	\textbf{Primary actor:}
	Person \\

	\textbf{Stakeholders and interests:} \\
	The person wants to prove their identity to access the features of the system that only an authenticated user can \\

	\textbf{Preconditions:} The person is not authenticated \\

	\textbf{Postconditions:}
	The person (now a user) has proven their identity to the system and has been recognized \\

	\textbf{Main success scenario:}
	\begin{enumerate}[itemsep=0pt, topsep=0pt]
		\item The person requests to authenticate with the system
		\item The system prompts the person to provide their email address and password
		\item The person submits the requested credentials
		\item The system confirms that the person has been successfully authenticated and provides account information, such as the username
	\end{enumerate}
	\vspace{1\baselineskip}

	\textbf{Extensions:} \\
	\hspace*{1em} 4a.1 The system informs the person that the authentication has failed and gives them the \\
	\hspace*{3em} option to either register a new user or try again \\[0.5em]
	\hspace*{2em} 4a.2a The person asks to try again \\[0.5em]
	\hspace*{2em} 4a.3a The use case starts again from step 1 \\[0.5em]
	\\
	\hspace*{2em} 4a.2b The person asks to register a new user \\[0.5em]
	\hspace*{2em} 4a.3b The use case \textbf{Register user} starts\\[0.5em]


	\textbf{Special Requirements:} None  \\
	
	\textbf{Technology and Data Variations List:} None  \\

	\textbf{Frequency of Occurrence:} Could be nearly continuous \\

	\textbf{Open issues:} None\\
\end{usecasebox}
\newpage

\begin{usecasebox}[title= Register user]
	\textbf{Scope:}
	GreenRoute \\

	\textbf{Level:}
	User goal \\

	\textbf{Primary actor:}
	Person \\

	\textbf{Stakeholders and interests:} \\
	The person wants to register their identity in the system to access the features that only an authenticated user can \\

	\textbf{Preconditions:} None \\

	\textbf{Postconditions:}
	The person has registered an identity in the system\\

	\textbf{Main success scenario:}
	\begin{enumerate}[itemsep=0pt, topsep=0pt]
		\item The person requests to register an identity with the system
		\item The system prompts the person to provide an email address and the same password twice
		\item The person submits the requested credentials
		\item The system emails the person a verification email
		\item The person verifies their email address within the time frame allowed
		\item The system confirms that the user has been successfully registered.
	\end{enumerate}
	\vspace{1\baselineskip}

	\textbf{Extensions:} \\
	\hspace*{1em} 4a.1 The system informs the person that the email they provided is not valid \\[0.5em]
	\hspace*{1em} 4a.2 The use case starts again from step 2 \\[0.5em]
	\\
	\hspace*{1em} 4b.1 The system informs the person that the two passwords they provided don't match \\[0.5em]
	\hspace*{1em} 4b.2 The use case starts again from step 2 \\[0.5em]
	\\
	\hspace*{1em} 4c.1 The system informs the person that the two passwords they provided are not valid \\[0.5em]
	\hspace*{1em} 4c.2 The use case starts again from step 2 \\[0.5em]
	\\
	\hspace*{1em} 5a.1 The person tries to verify their email address after the verification email has expired  \\[0.5em]
	\hspace*{1em} 5a.2 The system informs them of the error and prompts them to try to register again \\[0.5em]

	\textbf{Special Requirements:}
	\begin{itemize}[label=-,itemsep=0pt, topsep=0pt]
		\item The system should impose rules on what passwords are allowed that ensure all passwords are at least moderately secure
	\end{itemize}
	\vspace{1\baselineskip}
	
	\textbf{Technology and Data Variations List:} None \\

	\textbf{Frequency of Occurrence:} Could be nearly continuous \\

	\textbf{Open issues:} None\\
\end{usecasebox}
\newpage

\begin{usecasebox}[title= View possible routes]
	\textbf{Scope:}
	GreenRoute \\

	\textbf{Level:}
	User goal \\

	\textbf{Primary actor:}
	Person  or User\\

	\textbf{Stakeholders and interests:} \\
	The person wants the system to give them route recommendations tailored to their needs \\

	\textbf{Preconditions:} None \\

	\textbf{Postconditions:}
	The system has shown the actor routes that match the demands of the user\\

	\textbf{Main success scenario:}
	\begin{enumerate}[itemsep=0pt, topsep=0pt]
		\item The system provides the actor with map information
		\item The actor requires to see routes to a specific location within the map
		\item The system asks to be given a starting position
		\item The actor provides a starting position
		\item The system shows the actor what modes of transport are available and prompts them to choose a subset of them
		\item The actor selects the allowed modes of transport
		\item The system asks the actor if the routes should prioritize speed, eco friendliness or be balanced
		\item The actor selects an option
		\item The system gives the actor a list of possible routes with time estimates, distances traveled and estimated CO2 emissions	
	\end{enumerate}
	\vspace{1\baselineskip}
	
	\textbf{Extensions:} None\\

	\textbf{Special Requirements:} None \\
	
	\textbf{Technology and Data Variations List:} 
	\begin{itemize}[label=-,itemsep=0pt, topsep=0pt]
		\item The actor should be able to insert start and arrival locations by either specifying them manually or by using their phone's current GPS location 
	\end{itemize}
	\vspace{1\baselineskip}

	\textbf{Frequency of Occurrence:} Could be nearly continuous \\

	\textbf{Open issues:} None\\
\end{usecasebox}
\newpage

\subsection{Brief use cases}

\paragraph{Manage the route history:}
Users can look at the history of routes they have taken and have the option to delete parts or all of it.\\
\paragraph{Manage friends:}
Users can get a list of all their friends. For each of those friends, they can remove the friendship, invite them to a game challenge, look at their game attributes or block them.\\
\paragraph{Add a friend:}
Users can send friend requests to other users by specifying their username. The other user can accept the request, deny, or block the sender.\\
\paragraph{Manage blocked users:}
Users can get a list of all the users they blocked over time and have the option to unblock any them.\\
\paragraph{View the game status:}
Users can view all their own game attributes, such as points, challenges they won or badges they obtained and view rankings that compare them to their friends.\\
\paragraph{Manage the system:}
Administrators can see system wide information that is hidden from regular users, such as API usage and take system wide actions, such as temporarily blocking the creation of new accounts.\\
\paragraph{Manage administrators:}
Administrators can manage the permissions of administrators of lower rank, make a user an administrator of lower rank than them or remove administrator privileges entirely from administrators of lower rank. \\


\paragraph{Manage route history:}
Users can view the history of routes they have taken and may delete individual entries, entries within a time frame or clear the entire history.\\

\paragraph{Manage friends:}
Users can view a list of all their friends. For each friend, they may remove the friendship, invite the friend to a game challenge, view the friend’s game-related attributes, or block the friend.\\

\paragraph{Add a friend:}
Users can send friend requests to other users by specifying their username. The recipient may accept the request, decline it, or block the sender.\\

\paragraph{Manage blocked users:}
Users can view a list of all users they have blocked and may unblock any user at any time.\\

\paragraph{View game status:}
Users can view their own game-related attributes, such as points earned, challenges won, and badges obtained. They can also view rankings that compare their performance with that of their friends.\\

\paragraph{Manage the system:}
Administrators can access system-wide information that is not available to regular users, such as API usage statistics, and can perform system-wide actions, including temporarily disabling the creation of new accounts.\\

\paragraph{Manage administrators:}
Administrators can manage the permissions of administrators with a lower rank. This includes granting administrator privileges at a lower rank or revoking administrator privileges entirely from lower-ranked administrators.\\
